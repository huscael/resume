% !TEX TS-program = xelatex
% !TEX encoding = UTF-8 Unicode
% !Mode:: "TeX:UTF-8"

\documentclass{resume}
\usepackage{zh_CN-Adobefonts_external} % Simplified Chinese Support using external fonts (./fonts/zh_CN-Adobe/)
%\usepackage{zh_CN-Adobefonts_internal} % Simplified Chinese Support using system fonts
\usepackage{linespacing_fix} % disable extra space before next section
\usepackage{cite}

\begin{document}
\pagenumbering{gobble} % suppress displaying page number

\name{唐天舒}

\basicInfo{
  \email{tstang@zju.edu.cn} \textperiodcentered\ 
  \phone{13588076479} }
 
\section{\faGraduationCap\  教育背景}
\datedsubsection{\textbf{浙江大学}, 杭州, 浙江}{2020 -- 至今}
\textit{在读硕士研究生}\ 软件工程, 预计 2023 年 3 月毕业
\datedsubsection{\textbf{重庆邮电大学}, 重庆}{2016 -- 2020}
\textit{学士}\ 计算机科学与技术

\section{\faUsers\ 实习/项目经历}
\datedsubsection{超前预警设备云平台}{2021年6月 -- 至今}
\role{Java 后端开发}{浙大滨江研究院合作项目}
基于 SpringBoot 框架分别实现了设备通讯协议解析模块与云平台后台数据管理模块, 目前项目已投入 生产。带领项目团队参加华为鲲鹏应用创新大赛,获得了浙江赛区总决赛二等奖。\\ 
个人工作内容
\begin{itemize}
  \item 根据协议文档实现设备通讯协议解析, 部署开源 MQTT 服务器 EMQX 维护设备连接
  \item 使用 redis 作为缓存数据库,进行实时设备在线状态更新与查询,并通过 expire 指令动态设置设备
命令延迟下发
  \item 云平台 Mysql 数据库设备数据模块增删改查,使用了 Mybatis 作为持久层框架
  \item 使用 SpringBoot 中 AOP 功能实现接口通信数据日志打印, 并通过自定义注解结合 AOP 实现接口
权限控制
  \item 通过 JWT 实现了云平台用户验证
\end{itemize}

% Reference Test
%\datedsubsection{\textbf{Paper Title\cite{zaharia2012resilient}}}{May. 2015}
%An xxx optimized for xxx\cite{verma2015large}
%\begin{itemize}
%  \item main contribution
%\end{itemize}

\section{\faCogs\ IT 技能}
% increase linespacing [parsep=0.5ex]
\begin{itemize}[parsep=0.5ex]
  \item 编程语言: Java, Python, C
  \item 平台: Linux, 熟悉常用 linux 命令, 能够在 linux 上进行项目开发与部署,并使用 shell 脚本结合 CI/CD 服务进行自动运维
  \item 数据结构与算法: 可实现快速排序,归并排序等经典算法;熟悉链表,跳表,二叉搜索树等数据结 构
  \item JavaWeb框架:了解Spring框架设计理念,如IoC,AOP等特性,读过部分源码并了解其设计模式
  \item 数据库:熟悉关系型数据库 Mysql,有设计数据表经验,了解索引优化、SQL 查询优化以及底层
 存储引擎。熟悉非关系型数据库 Redis, 了解其基本数据类型的底层实现原理
\end{itemize}

\section{\faHeartO\ 获奖情况}
\datedline{\textit{银奖}, ACM-ICPC 大学生程序设计大赛四川省赛区}{2017 年4 月}
\datedline{\textit{特等奖}, 全国大学生英语竞赛总决赛}{2017 年4 月}
\datedline{\textit{二等奖}, 蓝桥杯全国软件和信息技术专业人才大赛全国总决赛}{2018 年10 月}
\datedline{\textit{第二名},  CCF 大学生计算机系统与程序设计竞赛西南赛区}{2019 年4 月}
\datedline{\textit{铜奖}, CCF 大学生计算机系统与程序设计竞赛}{2019 年10 月}
\datedline{CCF 全国优秀大学生}{2019 年10 月}
\datedline{国家奖学金}{2019 年12 月}

\section{\faInfo\ Miscellaneous}
% increase linespacing [parsep=0.5ex]
\begin{itemize}[parsep=0.5ex]
  \item 语言: 英语 - 熟练 (CET6 607)
\end{itemize}

%% Reference
%\newpage
%\bibliographystyle{IEEETran}
%\bibliography{mycite}
\end{document}


